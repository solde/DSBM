\documentclass{article}
\usepackage{hyperref}
\usepackage[catalan]{babel}
\selectlanguage{catalan}

\hypersetup{
    colorlinks,
    citecolor=black,
    filecolor=black,
    linkcolor=black,
    urlcolor=black
}

\title{DSBM}
\author{David Soldevila}

\begin{document}

\maketitle

\pagebreak

\tableofcontents

\pagebreak

\section{Introducció}

\subsection{Embedded system}

\begin{itemize}
	\item Funció espedífica
	\item Part d'un sistema més ampli
	\item Basats en micros
	\item Hadware limitat
	\item Qualitat i fiabilitat elevats
	\item Adequat per sistemes a temps real
\end{itemize}

\pagebreak

\section{Repàs de conceptes electrònics}

\subsection{Generadors}

\begin{itemize}
	\item Generació de tensió alterna
	\begin{itemize}
		\item Xarxa elèctrica, dinamo.
	\end{itemize}
	\item Generació de tensió continua
	\begin{itemize}
		 \item Bateries, piles, cel·les fotoelèctriques, fonts d'alimentació...
	\end{itemize}
\end{itemize}

\subsection{Conductors i aïllants}

Conductor: material que oposa pocs resitència al pas de corrent.

Aïllant: material que s'oposa al pas de la corrent.

Semiconductors: material que està entre conductors i aïllants i que permet controlar la conductivitat mitjançant l'adició dimpreses p(+) i N(-).

\subsection{Dispositius passius}

\begin{itemize}
	\item Resistències

$R = \frac{Va - Vb}{I}$

	\item Condensadors

$I_c = C \frac{dv_c}{dt}$; $V_c = \frac{1}{C} \int I_c dt$
	\begin{itemize}
		\item S'oposa a canvis ràpids de la tensió.
		\item Subministra corrent sense variar al sensivilitat.
		\item La impedància disminueix en funció de la freqüència.
	\end{itemize}
	\item Inductàncies

$V_l = -L \frac{dI_l}{dt}$: $I_c = \frac{1}{L} \int V_l dt$

	\begin{itemize}
		\item S'oposa als canvis ràpids de corrent.
		\item Subministra pics de tensio.
		\item La impedància augmenta en funció de la freqüència.
	\end{itemize}
\end{itemize}

\subsection{Lleis de kirchoff}
	\begin{itemize}
		\item Malla: Elemets d'un circuit connectats formant un llaç tencat.

$\sum_0^nV_i = 0 \Rightarrow V_{cc} = V_1 + V_2 + V_3 + ... V_n$

		\item Node: punt d'un circuit on es connecten dos o més elements.
	\end{itemize}

$\sum_0^nI_i = 0 \Rightarrow I_{in} = I_{out}$

\subsection{Ciucuit RC}

Composat per un generador i una Resistència.

\subsection{Càrrega i descàrrega d'un condensador}

$\tau$ és el component temporal que depen del producte RC.

$V_c(t) = V_{cc}(1-e^(\frac{-t}{RC}))$

\subsection{Diode}

Permet la circulació de corrent en només un sentit.

$V_d > V_f \Rightarrow On (I_f \neq 0)$

$V_d < V_f \Rightarrow Off (I_f = 0)$

Funcions:
\begin{itemize}
	\item Rectificadors
	\item Retalladors
	\item Protecció
\end{itemize}

\subsubsection{Diode Zenner}

A partir d'una certa tensio negativa (llindà de ruptura) deixa passar el corrent.

$|V_d| > |V_z| \Rightarrow On (I_f \neq 0)$

$|V_d| < |V_z| \Rightarrow Off (I_f = 0)$

\end{document}